\documentclass[lualatex,ja=standard,magstyle=real,12pt]{bxjsarticle}
\usepackage[backend=biber,sortcites=true,yearsuffix]{biblatex-japanese}
\addbibresource{test-japanese.bib}
\makeatletter
\renewcommand{\@makefnmark}{\hbox{}\hbox{\@textsuperscript{\normalfont\@thefnmark}}\hbox{}}
\makeatother
\begin{document}
文献\autocite{hoge2000,foobar1990}はこの分野で最も広く読まれている基礎文献であり,
大学院に入学するまでに必読である
\footnote{\cite{foobar1990}は長大な論文であり,和訳が単行本で出ている\cite{foobar1995}.}.
特に平安時代の文化との関わり\autocite[25]{hoge2000},
英語と日本語の言語学的関連からの考察\autocite[30--35]{hoge2000}は興味深い.
また,文献\autocite{hoge2001}は新たな分野を拓いた最初の論文であり,
当初の問題意識を知るうえで重要である.
\printbibliography
\end{document}
